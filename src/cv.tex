%!TEX TS-program = xelatex
%!TEX encoding = UTF-8 Unicode
%!TEX root
%
% The original template has been downloaded from:
% https://github.com/posquit0/Awesome-CV
%
% Author:
% Hervé Nivon <herve.nivon@gmail.com>
% https://www.linkedin.com/in/hervenivon/
%


%-------------------------------------------------------------------------------
% CONFIGURATIONS
%-------------------------------------------------------------------------------
% A4 paper size by default, use 'letterpaper' for US letter
\documentclass[9pt, a4paper]{awesome-cv}

% Additional package for timeline preview:
% - http://ctan.localhost.net.ar/macros/latex/contrib/chronology/chronology.pdf
% - https://tex.stackexchange.com/questions/385440/latex-chronology-change-end-of-timeline/385443
\usepackage{chronology}
\usepackage{tikz}
\usetikzlibrary{arrows}

% Configure page margins with geometry
\geometry{left=1.45cm, top=.8cm, right=1.45cm, bottom=1.8cm, footskip=.5cm}

% Specify the location of the included fonts
\fontdir[fonts/]


% Color for highlights
% Awesome Colors: awesome-emerald, awesome-skyblue, awesome-red, awesome-pink, awesome-orange
%                 awesome-nephritis, awesome-concrete, awesome-darknight
% \colorlet{awesome}{awesome-red}
% Uncomment if you would like to specify your own color
% \definecolor{awesome}{HTML}{DA291C} %redbird red
% \definecolor{awesome}{HTML}{FF9900} %AWS orange
\definecolor{awesome}{HTML}{439F97} %Scenario green

% Additional colors for text
% Uncomment if you would like to specify your own color
% \definecolor{darktext}{HTML}{414141}
% \definecolor{text}{HTML}{333333}
% \definecolor{graytext}{HTML}{5D5D5D}
% \definecolor{lighttext}{HTML}{999999}

% Set false if you don't want to highlight section with awesome color
\setbool{acvSectionColorHighlight}{true}

% If you would like to change the social information separator from a pipe (|) to something else
\renewcommand{\acvHeaderSocialSep}{\quad\textbar\quad}

%-------------------------------------------------------------------------------
%   PERSONAL INFORMATION
%   Comment any of the lines below if they are not required
%-------------------------------------------------------------------------------

% Available options: circle|rectangle,edge/noedge,left/right
\photo[circle,edge,left]{./src/images/profile.jpg}
\name{Hervé}{Nivon}
\position{CTO{\enskip\cdotp\enskip}Talent Developer{\enskip\cdotp\enskip}Computer Scientist{\enskip\cdotp\enskip}Solutions Architect{\enskip\cdotp\enskip}Business Angel}
\address{Meudon, France}
\mobile{+33 6 32 68 85 74}
\email{herve.nivon@gmail.com}
\linkedin{hervenivon}
\twitter{@hervenivon}
\github{hervenivon}
\stackoverflow{3012703}{herve}
% \gitlab{gitlab-id}
% \skype{skype-id}
% \reddit{reddit-id}
% \extrainfo{extra informations}

%\quote{``It’s not an experiment if you know it’s going to work." --- Jeff Bezos}
%\quote{``Everything fails, all the time." --- Werner Vogels}
\quote{``There is no compression algorithm for experience." --- Andy Jassy}
%\quote{``There is only one boss. The customer." --- Sam Walton}
%\quote{``Refuse to lower your standards to accommodate those who refuse to raise theirs." --- Mandy Hale}
%\quote{``Focus on the journey not the destination. Joy is found not in finishing an activity but in doing it." --- Greg Anderson}
%\quote{``Fail often so you can succeed sooner." --- Tom Kelley}
%\quote{``If you can't feed a team with tow pizzas, it's too large." --- Jeff Bezos}
%\quote{``The creative adult is the child who has survived." --- Ursula K. Le Guin}
%\quote{``Without passion, any rational person would give up." --- Steve Jobs}
%\quote{``The smartest companies don’t tell their employees how to innovate, they manage the chaos."\newline--- Eric Schmidt}


%-------------------------------------------------------------------------------
\begin{document}

% Print the header with above personal information
% Give optional argument to change alignment(C: center, L: left, R: right)
\makecvheader[R]

% Print the footer with 3 arguments(<left>, <center>, <right>)
% Leave any of these blank if they are not needed
\makecvfooter
  {\today}
  {Hervé Nivon~~~·~~~Curriculum Vitae}
  {\thepage}

%-------------------------------------------------------------------------------
%	CV/RESUME CONTENT
%	Each section is imported separately, open each file in turn to modify content
%-------------------------------------------------------------------------------
%-------------------------------------------------------------------------------
%   SECTION TITLE
%-------------------------------------------------------------------------------
\cvsection{Summary}

%-------------------------------------------------------------------------------
%   CONTENT
%-------------------------------------------------------------------------------
\begin{cvparagraph}

%---------------------------------------------------------
CTO and Co-Founder of Scenario, a platform that assists indie developers and studios in effortlessly creating stunning, custom game assets with speed and artistic consistency.

Previously Senior Startup Solutions Architect at Amazon Web Services, aiding hundreds of startups in effectively onboarding, growing, and scaling.

Former CTO of a company specializing in generating business insights from drone imagery using machine learning and big data analytics.

Over ten years of experience, he has worked on large programs for Fortune\nobreakspace500 companies and led innovation for Accenture’s Resources France.

Holds a Master’s degree in Computer Science and Software Engineering, during which he taught algorithmics and programming languages.

Enjoys fine food, cocktails, and wine tasting with family and friends, blacksmithing, collecting and creating pop art, and engaging in activities with his son, especially collecting and building Lego sets.
\end{cvparagraph}

%---------------------------------------------------------
\begin{cventries}
    {
        % This section depends on chronology package
        % Documentation is here:
        % - http://ctan.localhost.net.ar/macros/latex/contrib/chronology/chronology.pdf
        % - https://www.complang.tuwien.ac.at/doc/texlive-doc/latex/chronology/chronology.pdf
        %
        % \decimaldate{day}{month}{year}

        \definecolor{redbird}{HTML}{DA291C}
        \definecolor{airware}{HTML}{343641}
        \definecolor{aws}{HTML}{FF9900}
        \definecolor{scenario}{HTML}{439F97}
        \tikzset{>=|}

        % Default chronoloygy tikzset settings
        % \tikzset{%
        %     ,chronevent/.style={fill=black,draw=none,opacity=0.5}
        %     ,chronlabel/.style={opacity=1}
        %     ,chrontickslabel/.style={chronlabel}
        %     ,chroneventlabel/.style={chronlabel}
        %     ,eventlabel/.style={chroneventlabel,anchor=south west,yshift=.2\unit,rotate=45}
        %     ,flippedeventlabel/.style={chroneventlabel,anchor=north west,yshift=-.2\unit,rotate=-45}
        %     }
        %
        % \tikzset{chronevent/.style={fill=black,draw=none,opacity=0.5,line width=0.1pt}}

        % WORKING Chronology example:
        %
        % \begin{chronology}[3]{1984}{2018}{0.84\textwidth}
        %   \event{1984}{\color{awesome}birth}
        %   % \tikzset{chronevent/.style={fill=black,draw=none,opacity=0.5}}
        %   % \tikzset{chronevent/.style={fill=testlightgray,draw=none,opacity=0.5}}
        %   \tikzset{chronevent/.style={fill=awesome,draw=none,opacity=0.5}}
        %   \event[1985]{1986}{two}
        %   \tikzset{chronevent/.style={fill=black,draw=none,opacity=0.5}}
        %   \event{\decimaldate{25}{12}{2001}}{three}
        % \end{chronology}

        % \begin{chronology}[3]{2007}{2021}{0.85\textwidth}[\textwidth]
        % Do not start at the true beginning of the experience, not enough space on the chart
        \begin{chronology}[3]{2011}{2023}{0.85\textwidth}[\textwidth]
            % Accenture milestones
            \event{\decimaldate{13}{7}{2011}}{\descriptionstyle{Historical System Replacement}}
            \event{\decimaldate{1}{5}{2012}}{\descriptionstyle{Architecture Design of Strategic Project}} % Box Engie
            \event{\decimaldate{1}{9}{2013}}{\descriptionstyle{Indian Delivery Center Deployment}}
            \event{\decimaldate{1}{5}{2014}}{\descriptionstyle{1st Place of Innovation Contest}}
            \event{\decimaldate{1}{12}{2014}}{\descriptionstyle{PoC for Drone data processing @ scale}}
            \event{\decimaldate{15}{6}{2015}}{\descriptionstyle{20000+ man.days delivered}}
            \event{\decimaldate{1}{9}{2015}}{\descriptionstyle{Innovation lead}}

            \tikzset{chronevent/.style={fill=redbird,draw=none,opacity=0.5}}
            \event{\decimaldate{1}{12}{2013}}{\descriptionstyle{Meet Redbird}}
            \tikzset{chronevent/.style={fill=black,draw=none,opacity=0.5}}

            % Redbird milestones
            \tikzset{chronevent/.style={fill=redbird,draw=none,opacity=0.7}}
            \event{\decimaldate{1}{7}{2016}}{\descriptionstyle{1st Serverless microservice}}
            \event{\decimaldate{1}{11}{2016}}{\descriptionstyle{Becoming a global company}}
            \event{\decimaldate{1}{5}{2017}}{\descriptionstyle{Breathtaking Deep Learning Results}}
            \event{\decimaldate{1}{9}{2017}}{\descriptionstyle{20+ serverless microservices}}
            \event{\decimaldate{31}{12}{2017}}{\descriptionstyle{SOTA on Aerial Imagery recognition}}
            \event{\decimaldate{1}{10}{2018}}{\descriptionstyle{2nd sell of company}}
            \tikzset{chronevent/.style={fill=black,draw=none,opacity=0.5}}

            % AWS milestones
            \tikzset{chronevent/.style={fill=aws,draw=none,opacity=0.7}}
            \event{\decimaldate{1}{4}{2020}}{\descriptionstyle{1st EMEA Activate Onboarding}}
            \event{\decimaldate{30}{9}{2020}}{\descriptionstyle{Major Partnership signed}}
            \event{\decimaldate{1}{4}{2021}}{\descriptionstyle{Dozen public speaking achieved}}
            \event{\decimaldate{2}{7}{2021}}{\descriptionstyle{400+ Startups engaged in 1:1}}
            \tikzset{chronevent/.style={fill=black,draw=none,opacity=0.5}}

            % Scenario milestones
            \tikzset{chronevent/.style={fill=scenario,draw=none,opacity=0.7}}
            \event{\decimaldate{1}{12}{2022}}{\descriptionstyle{Beta Scenario GenAI}}
            \event{\decimaldate{9}{4}{2023}}{\descriptionstyle{5M Images Generated}}
            \event{\decimaldate{1}{2}{2024}}{\descriptionstyle{2.2K Daily Active Users; 100K Custom Models}}
            \tikzset{chronevent/.style={fill=black,draw=none,opacity=0.5}}

            %Periods
            %Set periods under the timeline
            \tikzset{eventlabel/.style={chroneventlabel,anchor=south,yshift=-0.85\unit,rotate=0}}
            %Set a light default for periods
            \tikzset{chronevent/.style={fill=black,draw=none,opacity=0.2}}

            \event[\decimaldate{1}{1}{2011}]{\decimaldate{31}{8}{2012}}{\descriptionstyle{Consultant}}
            \event[\decimaldate{31}{8}{2012}]{\decimaldate{8}{1}{2016}}{\descriptionstyle{Manager}}

            \tikzset{chronevent/.style={fill=redbird,draw=none,opacity=0.2}}
            \event[\decimaldate{16}{1}{2016}]{\decimaldate{26}{5}{2019}}{\descriptionstyle{Redbird/Airware/Delair}}
            \tikzset{chronevent/.style={fill=black,draw=none,opacity=0.5}}

            \tikzset{chronevent/.style={fill=aws,draw=none,opacity=0.2}}
            \event[\decimaldate{27}{5}{2019}]{\decimaldate{2}{7}{2021}}{\descriptionstyle{AWS}}
            \tikzset{chronevent/.style={fill=black,draw=none,opacity=0.5}}

            \tikzset{chronevent/.style={fill=scenario,draw=none,opacity=0.2}}
            \event[\decimaldate{2}{7}{2021}]{\decimaldate{1}{3}{2024}}{\descriptionstyle{Scenario}}
            \tikzset{chronevent/.style={fill=black,draw=none,opacity=0.5}}
        \end{chronology}
        \vspace{0.10cm}
    }
\end{cventries}

%-------------------------------------------------------------------------------
%	SECTION TITLE
%-------------------------------------------------------------------------------
\cvsection{Skills}

%-------------------------------------------------------------------------------
%	CONTENT
%-------------------------------------------------------------------------------
\begin{cvskills}

%---------------------------------------------------------
  \cvskill
    {Management} % Category
    {People Development, Program and Project management, Risk assesment, Financials, Quality Assurance} % Skills

%---------------------------------------------------------
  \cvskill
    {Business intelligence} % Category
    {Product Design, Process Design, IT Strategy, Business Analysis, Business Transformation, Public speaking} % Skills

%---------------------------------------------------------
  \cvskill
    {Business} % Category
    {Gaming, Software Systems Architecture Engineering, Utilities and Energy} % Skills

%---------------------------------------------------------
  \cvskill
    {Cloud computing} % Category
    {Hundreds of production workloads: API, Models Training and serving, Analytics, Serverless Computing and Containers} % Skills

% %---------------------------------------------------------
%   \cvskill
%     {Programming} % Category
%     {\textit{Used or managed in production}: Node.js, Javascript, Python, Keras, Tensorflow, C/C++, Go, Ruby, Java} % Skills

% %---------------------------------------------------------
%   \cvskill
%     {Web} % Category
%     {\textit{Used or managed in production}: HTML5, React, Redux, JQuery, Angular} % Skills

%---------------------------------------------------------
  \cvskill
    {Languages} % Category
    {English, French} % Skills

%---------------------------------------------------------
\end{cvskills}

\input{src/cv/education.tex}
% Adapted from https://tex.stackexchange.com/questions/82727/create-a-ring-diagram-in-tex/82729#82729

% The wheelchart macro
\newcommand{\wheelchart}[3]{
    \def\innerradius{#2}%
    \def\outerradius{#1}%

    % Calculate total
    \pgfmathsetmacro{\totalnum}{0}
    \foreach \value/\colour/\name in {#3} {
        \pgfmathparse{\value+\totalnum}
        \global\let\totalnum=\pgfmathresult
    }

    \begin{tikzpicture}

      % Calculate the thickness and the middle line of the wheel
      \pgfmathsetmacro{\wheelwidth}{\outerradius-\innerradius}
      \pgfmathsetmacro{\midradius}{(\outerradius+\innerradius)/2}

      % Rotate so we start from the top
      \begin{scope}[rotate=90]

      % Loop through each value set. \cumnum keeps track of where we are in the wheel
      \pgfmathsetmacro{\cumnum}{0}
      \foreach \value/\colour/\name in {#3} {
            \pgfmathsetmacro{\newcumnum}{\cumnum + \value/\totalnum*360}

            % Calculate the percent value
            \pgfmathsetmacro{\percentage}{\value/\totalnum*100}
            % Calculate the mid angle of the colour segments to place the labels
            \pgfmathsetmacro{\midangle}{-(\cumnum+\newcumnum)/2}

            % This is necessary for the labels to align nicely
            \pgfmathparse{
               (-\midangle<180?"west":"east")
            } \edef\textanchor{\pgfmathresult}
            \pgfmathsetmacro\labelshiftdir{1-2*(-\midangle>180)}

            % Draw the color segments. Somehow, the \midrow units got lost, so we add 'pt' at the end. Not nice...
            \fill[\colour] (-\cumnum:\outerradius) arc (-\cumnum:-(\newcumnum):\outerradius) --
            (-\newcumnum:\innerradius) arc (-\newcumnum:-(\cumnum):\innerradius) -- cycle;

            % Draw the data labels with percentages
            % \draw  [*-,thin] node [append after command={(\midangle:\midradius pt) -- (\midangle:\outerradius + 1ex) -- (\tikzlastnode)}] at (\midangle:\outerradius + 1ex) [xshift=\labelshiftdir*0.5cm,inner sep=0pt, outer sep=0pt, ,anchor=\textanchor]{\name: \pgfmathprintnumber{\percentage}\%};
            % Draw the data labels without percentages
            \draw  [*-,thin] node [append after command={(\midangle:\midradius pt) -- (\midangle:\outerradius + 1ex) -- (\tikzlastnode)}] at (\midangle:\outerradius + 1ex) [xshift=\labelshiftdir*0.5cm,inner sep=0pt, outer sep=0pt, ,anchor=\textanchor]{\name};

            % Set the old cumulated angle to the new value
            \global\let\cumnum=\newcumnum
        }

      \end{scope}
    \end{tikzpicture}
}

%-------------------------------------------------------------------------------
%   SECTION TITLE
%-------------------------------------------------------------------------------
\cvsection{Work Life Balance}

%-------------------------------------------------------------------------------
%   CONTENT
%-------------------------------------------------------------------------------
\begin{cventries}

\vspace{0.23cm}

% Usage: \wheelchart{<value1>/<colour1>/<label1>, ...}
\wheelchart{1.3cm}{0.80cm}{
    25/awesome/\descriptionstyle{Ideation \& Problem Solving},
    30/awesome!90/\descriptionstyle{Prototyping \& Development},
    15/awesome!80/\descriptionstyle{Training},
    5/awesome!40/\descriptionstyle{Blacksmithing, art making \& collecting},
    5/awesome!50/\descriptionstyle{Epicurism with family and friend},
    10/awesome!60/\descriptionstyle{Playing with my son, building Lego sets},
    10/awesome!80/\descriptionstyle{Startups Advisory}
}

%---------------------------------------------------------
\end{cventries}

%-------------------------------------------------------------------------------
%   SECTION TITLE
%-------------------------------------------------------------------------------
\cvsection{Strenghts}

%-------------------------------------------------------------------------------
%   CONTENT
%-------------------------------------------------------------------------------
\begin{cventries}

  \cvtag{\descriptionstyle{Innovative Leader}}
  \cvtag{\descriptionstyle{Customer Obsessed}}
  \cvtag{\descriptionstyle{Hardworker}}
  \cvtag{\descriptionstyle{Bridge Maker}}
  \cvtag{\descriptionstyle{Creative Problem Solver}}
  \cvtag{\descriptionstyle{Retentless Curious Learner}}

  %---------------------------------------------------------
\end{cventries}
\newpage
%-------------------------------------------------------------------------------
%   SECTION TITLE
%-------------------------------------------------------------------------------
\cvsection{Work Experience}


%-------------------------------------------------------------------------------
%   CONTENT
%-------------------------------------------------------------------------------
\begin{cventries}

%---------------------------------------------------------
\cventry
{CTO} % Job title
{Scenario.com} % Organization
{Paris} % Location
{July 2021 - Present} % Date(s)
{
  \begin{cvitems} % Description(s) of tasks/responsibilities
    \item {Support onboarding and growth of startups on Amazon Web Services, definition of the early stage market strategy.}
    \item {Primary responsibilities: customer best interest, CTO in residence, Content Creation, Trainings, Public speaking.}
    \item {Major tasks accomplished:}
    \begin{cvsubitems}
      \item {Helped 300+ startups grow on AWS,}
      \item {Set the EMEA Activate Onboarding sessions and its associated early startup journey, 1000+ startup onboarded,}
      \item {Tens of public speaking, workshops and immersion days sessions leading,}
      \item {Blog post, immersion day and workshops content creation,}
      \item {Initiate a major and unique partnership between AWS and a Startup.}
    \end{cvsubitems}
    \item {Lessons learned:}
    \begin{cvsubitems}
      \item {Amazon culture,}
      \item {Embrassing cloud has a strong tendancy to accelerate startup growth.}
    \end{cvsubitems}
  \end{cvitems}
}

%---------------------------------------------------------
  \cventry
    {Senior Startup Solutions Architect} % Job title
    {Amazon Web Services} % Organization
    {Paris} % Location
    {May 2019 - July 2021} % Date(s)
    {
      \begin{cvitems} % Description(s) of tasks/responsibilities
        \item {Support onboarding and growth of startups on Amazon Web Services, definition of the early stage market strategy.}
        \item {Primary responsibilities: customer best interest, CTO in residence, Content Creation, Trainings, Public speaking.}
        \item {Major tasks accomplished:}
        \begin{cvsubitems}
          \item {Helped 300+ startups grow on AWS,}
          \item {Set the EMEA Activate Onboarding sessions and its associated early startup journey, 1000+ startup onboarded,}
          \item {Tens of public speaking, workshops and immersion days sessions leading,}
          \item {Blog post, immersion day and workshops content creation,}
          \item {Initiate a major and unique partnership between AWS and a Startup.}
        \end{cvsubitems}
        \item {Lessons learned:}
        \begin{cvsubitems}
          \item {Amazon culture,}
          \item {Embrassing cloud has a strong tendancy to accelerate startup growth.}
        \end{cvsubitems}
      \end{cvitems}
    }

%---------------------------------------------------------
  \cventry
    {Director of Engineering} % Job title
    {Delair} % Organization
    {Paris, Toulouse} % Location
    {Oct. 2018 - May 2019} % Date(s)
    {
      \begin{cvitems} % Description(s) of tasks/responsibilities
        \item {Help merged the Airware assets into the Delair's platform.}
        \item {Primary responsibilities: Product Management, Data Science and Machine Learning integration, Engineers management.}
        \item {Major tasks accomplished:}
        \begin{cvsubitems}
          \item {Able to support all former Airware's customers,}
          \item {Perpetuate the Redbird product.}
        \end{cvsubitems}
        \item {Lessons learned: Re-inventing the wheel is a business killer.}
      \end{cvitems}
    }

%---------------------------------------------------------
  \cventry
    {Director of Engineering, then CTO} % Job title
    {Airware} % Organization
    {Paris, San Francisco} % Location
    {Sep. 2016 - Oct. 2018} % Date(s)
    {
      \begin{cvitems} % Description(s) of tasks/responsibilities
        \item {Support the transformation of a French startup into a multinational company.}
        \item {Primary responsibilities: Product Management, Data Science and Machine Learning Roadmap definition, Platform architecture, Engineers management.}
        \item {Major tasks accomplished:}
        \begin{cvsubitems}
          \item {Machine Learning workloads operating in production,}
          \item {Architecture principles defined with Redbird leveraged at the global level.}
        \end{cvsubitems}
        \item {Lessons learned:}
        \begin{cvsubitems}
          \item {Machine Learning applications are more than ample. We have just scratched the surface of their capabilities,}
          \item {Cloud-native services empower development teams to do more with less.}
        \end{cvsubitems}
      \end{cvitems}
    }

%---------------------------------------------------------
  \cventry
    {Vice President of Engineering, Acting CTO} % Job title
    {Redbird} % Organization
    {Paris} % Location
    {Jan. 2016 - Sep. 2016} % Date(s)
    {
      \begin{cvitems} % Description(s) of tasks/responsibilities
        \item {Transform an MVP into a viable and scalable product, developing advanced partnerships and new features.}
        \item {Primary responsibilities:} % Turning business vision into actionable technical strategy
        \begin{cvsubitems}
          \item {Architecture and Roadmap definition,}
          \item {Product Management \& 3rd parties modules partnership development,}
          \item {AWS team onboarding (from traditional databases and workloads to cloud-native services).}
        \end{cvsubitems}
        \item {Major tasks accomplished:}
        \begin{cvsubitems}
          \item {Working highly-available and fault tolerant architecture,}
          \item {Airware Acquisition Due Diligence,}
          \item {AWS Selection at the Startup Corner for Paris Summit.}
        \end{cvsubitems}
        \item {Lessons learned:}
        \begin{cvsubitems}
          \item {The world of startups fits me like a glove,}
          \item {Working with cutting-edge technologies thrills me,}
          \item {Finding my way off the beaten paths wakes me up in the morning.}
        \end{cvsubitems}
      \end{cvitems}
    }

%---------------------------------------------------------
  \cventry
    {Manager, Innovation Program lead for Resource France} % Job title
    {Accenture} % Organization
    {Paris Area} % Location
    {Jul. 2015 - Jan. 2016} % Date(s)
    {
      \begin{cvitems} % Description(s) of tasks/responsibilities
        \item {Major tasks accomplished:}
        \begin{cvsubitems}
          \item {Transformation program definition: spreading innovation through the Resources France group, including training selection,}
          \item {People acculturation and Network animation,}
          \item {``Paris Innovation Hub": offering definition, built a 20+ opportunities pipeline,}
          \item {Screened 10+ startups for future Accenture deals partnerships.}
        \end{cvsubitems}
        \item {Lessons learned:}
        \begin{cvsubitems}
          \item {All aspects of innovation thrill me,}
          \item {Train and experienced UX design, design thinking, lean startup, business model canvas.}
        \end{cvsubitems}
      \end{cvitems}
    }

%---------------------------------------------------------
  \cventry
    {Manager, Program Lead} % Job title
    {Accenture, for GRTgaz} % Organization
    {Paris Area} % Location
    {May 2012 - Jul. 2015} % Date(s)
    {
      \begin{cvitems} % Description(s) of tasks/responsibilities
        \item {Scope: 10+ applications in best of breed technologies, 60+ FTE on average (120+ max), 2 delivery centers,}
        \item {Features: Capacities Booking \& Product Management, Pricing, Measure, Allocation, Meter referential, CRM, Billing \& Invoicing, Gas Quality Analysis \& Simulation, Business Processes Supervision, Publication \& Alert Processing.}
        \item {Primary responsibilities: Transition lead, Finances management, Quality assurance, Risk, Program Management, Application Delivery, Proof of Concepts developments, Infrastructure Outsourcing.}
        \item {Major tasks accomplished:}
        \begin{cvsubitems}
          \item {Transition from a set of different sub-contractors to local developments, near-shore (Nantes) and offshore (Mumbai, India),}
          \item {AWS cloud implementation: development and staging environments,}
          \item {20000+ man.days, hundreds of releases, 98\%+ green SLAs, while maintaing net margin.}
        \end{cvsubitems}
        \item {Lessons learned:}
        \begin{cvsubitems}
          \item {Managing large teams in different offices --- including delivery centers on different time zones, different languages,}
          \item {Public Cloud is disruptive, enables new use cases, unleashes creativity. With a major drawback: requires additional and strong cost controls.}
        \end{cvsubitems}
      \end{cvitems}
    }

%---------------------------------------------------------
  \cventry
    {Consultant, Solution Architect} % Job title
    {Accenture, for GDF Suez} % Organization
    {Paris Area} % Location
    {Nov. 2009 - Sep. 2011} % Date(s)
    {
      \begin{cvitems} % Description(s) of tasks/responsibilities
        \item {Primary responsibilities:}
        \begin{cvsubitems}
          \item {Studying and writing solutions involving all impacted teams,}
          \item {Architectures and Roadmaps definitions and scenario sizing and scheduling.}
        \end{cvsubitems}
        \item {Major tasks accomplished:}
        \begin{cvsubitems}
          \item {Architecture weighing document exposed at company council for the replacement of SAP and Selligent with Salesforce,}
          \item {Application design for energy consumption and optimization B2C system,}
          \item {Architecture scenarios definition (different budgets, schedules, and impacts on the ecosystem).}
        \end{cvsubitems}
        \item {Lessons learned: Diplomacy is part of the Solution Architecture work.}
      \end{cvitems}
    }

%---------------------------------------------------------
  \cventry
    {Consultant, System Analyst for internal GDF Suez team} % Job title
    {Accenture, for GDF Suez} % Organization
    {Paris Area} % Location
    {Jan. 2009 - Sep. 2011} % Date(s)
    {
      \begin{cvitems} % Description(s) of tasks/responsibilities
        \item {Scope: SAP IS-U / CRM Utilities information system. 10 million customers, 3 thousand users, 40 million bills a year.}
        \item {Primary responsibilities:}
        \begin{cvsubitems}
          \item {Product manager: high-level business requirements gathering, solution design, sizing, general design writing,}
          \item {Support and monitoring of all down-stream teams.}
        \end{cvsubitems}
        \item {Major tasks accomplished:}
        \begin{cvsubitems}
          \item {Delivery of numerous features: Price determination, CRM, Machine-Machine Interfaces, Meter Reading \& Integration orchestrator, Consumption Extrapolation, Billing and Invoicing, Budget Billing Plan Subscription, Dunning,}
          \item {Automatic Invoice validation tool development,}
          \item {Utilities industry vulgarization and Accenture Resource team training.}
        \end{cvsubitems}
      \end{cvitems}
    }

%---------------------------------------------------------
  \cventry
    {Analyst} % Job title
    {Accenture, for Gaz De France} % Organization
    {Paris Area} % Location
    {Jan. 2007 - Dec. 2008} % Date(s)
    {
      \begin{cvitems} % Description(s) of tasks/responsibilities
        \item {Scope: SAP IS-U / CRM Utilities information system. 10 million customers, 3 thousand users, 40 million bills a year.}
        \item {Primary responsibilities:}
        \begin{cvsubitems}
          \item {Functional and technical requirements analyze, Solution design and costs determination,}
          \item {Planning, redaction and tests executions,}
          \item {Monitoring \& support of 60+ interfaces,}
        \end{cvsubitems}
        \item {Major tasks accomplished:}
        \begin{cvsubitems}
          \item {Multiple application designs and implementations related to machine-machine interfaces, pricing, billing and invoicing,}
          \item {Maintenance of production workloads, emergency solving,}
          \item {Business expertise acquisition.}
        \end{cvsubitems}
      \end{cvitems}
    }

%---------------------------------------------------------
  \cventry
    {Web developer} % Job title
    {Arcorp, Idoine, Satis (Web agency and software editor)} % Organization
    {Paris} % Location
    {Sep. 2005 - Dec. 2005} % Date(s)
    {
      \begin{cvitems} % Description(s) of tasks/responsibilities
        \item {Web application development for the FSE (Fonds Social Européen).}
      \end{cvitems}
    }

%---------------------------------------------------------
\end{cventries}

%-------------------------------------------------------------------------------
%   SECTION TITLE
%-------------------------------------------------------------------------------
\cvsection{Extracurricular Activities}

%-------------------------------------------------------------------------------
%   CONTENT
%-------------------------------------------------------------------------------
\begin{cventries}

%---------------------------------------------------------
  \cventry
    {Manager @ Accenture} % Job title
    {Intrapreuneurship experience: Video \& Image Analysis as a Service} % Organization
    {Paris Area} % Location
    {Dec. 2013 - Jul. 2015} % Date(s)
    {
      \begin{cvitems} % Description(s) of tasks/responsibilities
        \item {Business plan definition: Accenture provides a sclable infrastructure used by drone operators and service providers,}
        \item {Platform UX prototyping. Proof of Concept development: data processing using GPU is possible in the cloud, the product can scale.}
        \item {Lessons learned: innovation thrills me and never doing the same thing is key to me. I do love to deal with new technology.}
      \end{cvitems}
    }

%---------------------------------------------------------
  \cventry
    {iOS Developer} % Job title
    {Purring Box (Mobile Application)} % Organization
    {Paris Area} % Location
    {Jan. 2009 - Dec. 2009} % Date(s)
    {
      \begin{cvitems} % Description(s) of tasks/responsibilities
        \item {Product definition, design, development and deployment of a mobile application levraging purring therapy.}
        \item {Lessons learned: it can be easy to learn a new eco-system; getting an application from unknown to famous is the piece de resistance.}
      \end{cvitems}
    }

%---------------------------------------------------------
  \cventry
    {Teacher Assistant (20 tutors / 300 students), Web developer} % Job title
    {EPITA} % Organization
    {Paris} % Location
    {Jan. 2006 - Aug. 2006} % Date(s)
    {
      \begin{cvitems} % Description(s) of tasks/responsibilities
        \item {In charge of practical courses and lectures (Object Oriented Languages, Scripting, Design \& Design Pattersn, Unix).}
        \item {Internationalization of a social web site that manages ideas and initiatives development related to ecology and environment saving.}
      \end{cvitems}
    }

%---------------------------------------------------------
\end{cventries}

\input{src/cv/honors.tex}

% Will remain unwritten for the moment:
% \input{src/cv/presentation.tex}
% \input{src/cv/writing.tex}

%-------------------------------------------------------------------------------
\end{document}
